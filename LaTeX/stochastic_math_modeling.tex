\documentclass[11pt]{report}

\usepackage{a4wide}
\usepackage[T2A]{fontenc}
\usepackage[utf8]{inputenc}
\usepackage[main=russian, english]{babel}
\usepackage{graphicx}
\usepackage{amsmath}
\usepackage{amsthm}
\usepackage{amssymb}
\usepackage{amsfonts}
\usepackage{mathtools}
\usepackage{cases}
\usepackage{hyperref}
\usepackage{indentfirst}
\usepackage{float}
\usepackage{subcaption}
%\usepackage[nottoc,numbib]{tocbibind}

\graphicspath{{./}}

\begin{document}

\thispagestyle{empty}

\begin{center}
\ \vspace{-3cm}

\includegraphics[width=0.5\textwidth]{msu.eps}\\
{\scshape Московский государственный университет имени М.~В.~Ломоносова}\\
Факультет вычислительной математики и кибернетики\\
Кафедра системного анализа

\vfill

{\LARGE Отчёт по практикуму}

\vspace{1cm}

{\Huge\bfseries <<Стохастический анализ и моделирование>>}
\end{center}

\vspace{1cm}

\begin{flushright}
  \large
  \textit{Студент 415 группы}\\
  А.\,А.~Пилюшенок

  \vspace{5mm}

  \textit{Руководитель практикума}\\
  д.ф.-м.н., профессор С.\,Н.~Смирнов
\end{flushright}

\vfill

\begin{center}
Москва, 2024
\end{center}

% Custom Commands start here
%\let\oldref\ref
%\renewcommand{\ref}[1]{(\oldref{#1})}
\renewcommand{\phi}{\varphi}
\renewcommand{\sh}{\operatorname{sh}}
\renewcommand{\phi}{\varphi}
\renewcommand{\tilde}{\widetilde}
\newcommand{\const}{\operatorname{const}}
\newcommand{\atan}[1]{\operatorname{arctg}{#1}}
\newcommand{\mo}{\mathbb{E}}
\newcommand{\di}{\mathbb{D}}

\newtheorem{definition}{Определение}

\newpage % ToC starts here

\tableofcontents

\newpage % Main document starts here

\chapter{Первая часть практикума}

\section{Задание 1}

\section{Задание 2}

\section{Задание 3}

\section{Задание 4}

\section{Задание 5}

\section{Задание 6}

\chapter{Вторая часть практикума}

\section{Задание 7}

\subsection{Условие}

\begin{enumerate}
\item Методом случайного поиска найти минимальное значение функции $f$ на множестве $A = \{(x_1,x_2):~x_1^2+x_2^2\leqslant 1\}$, где
$$
f(x_1,x_2) = x_1^3\sin\frac{1}{x_1} + 10x_1x_2^4\cos\frac{1}{x_2},\quad x_1,x_2\neq0.
$$
При $x_1=0$ или $x_2=0$ функция доопределяется по непрерывности.

\item Методом имитации отжига найти минимальное значение функции Розенброка $g$ в пространстве $\mathbb{R}^2$, где
$$
g(x) = (x_1-1)^2 + 100(x_2-x_1^2)^2.
$$

\item Оценить точность и сравнить результаты со стандартными методами оптимизации.
\end{enumerate}

\subsection{Метод случайного поиска}

Метод случайного поиска минимального значения функции $f(x)$ на множестве $A$:
\begin{enumerate}
\item Смоделировать выборку размера $n$ точек $x\sim\mathrm{U}\{A\}$ (равномерно распределены на множестве A).
\item Выбрать ту реализацию случайной величины, на которой достигается наименьшее значение.
\end{enumerate}

Данная функция $f(x)=x_1^3\sin\frac{1}{x_1}+10x_1x_2^4\cos\frac{1}{x_2}$ обладает свойствами
\begin{enumerate}
\item $f(x_1,-x_2)=f(x_1,x_2)$ (чётность по $x_2$).
\item $f(-x_1,0)=f(x_1,0)=x_1^3\sin\frac{1}{x_1}$.
\item $f(0,0)=0$.
\end{enumerate}

Следовательно, если $x^*=(x_1^*,x_2^*)$ доставляет минимум функции, то $(x_1^*,-x_2^*)$ также доставляет минимум. \
Минимум меньше нуля, ведь $f(0.123, 0.123) < 0$ (проверяется численно). \
Следовательно, функция $f$ имеет хотя бы две точки глобального минимума.

По условию задачи множество $A$ - круг единичного радиуса на плоскости $(x_1,x_2)$. Пусть $x=(x_1,x_2)\sim\mathrm{U}(A)$. По определению для любого борелевского множества $M$ выполняется ($\mu$ - мера Лебега на плоскости)
\begin{multline*}
\mathbb{P}\left( (x_1,x_2)\in M \right) = \dfrac{\mu (M)}{\mu (A)} = \dfrac{1}{\pi}\iint\limits_M dt_1dt_2 = \begin{vmatrix}
t_1 = r\cos\alpha \\
t_2 = r\sin\alpha
\end{vmatrix} =\\
= \dfrac{1}{\pi}\iint\limits_M rdrd\alpha = \iint\limits_M d(r^2)d\left(\frac{\alpha}{2\pi}\right).
\end{multline*}

Следовательно, будем моделировать
$$
\begin{cases}
x_1 = \sqrt{\omega}\cos\alpha \\
x_2 = \sqrt{\omega}\sin\alpha
\end{cases}
$$
где $\omega\sim\mathrm{U}[0,1], \alpha\sim\mathrm{U}[0,2\pi]$.




\subsection{Метод имитации отжига}

\subsection{Сравнение методов со стандартными методами оптимизации}

\subsubsection{Метод случайного поиска}

Пусть $\varepsilon>0$ - наперед заданная точность вычислений. Пусть $x^*=(x_1^*,x_2^*)$ доставляет минимум функции $f(x_1,x_2)$. \
Оценим точность работы алгоритма при помощи многомерной теоремы Лагранжа
$$
|f(x) - f(x^*)| \leqslant \max\limits_A 
\sqrt{\left(\dfrac{\partial f}{\partial x_1}\right)^2 + \left(\dfrac{\partial f}{\partial x_2}\right)^2} \cdot || x - x^* ||.
$$

Справедливы следующие оценки
$$
\begin{aligned}
\left|\dfrac{\partial f}{\partial x_1}\right| &= \left| 3x_1^2\sin\frac{1}{x_1} - x_1\cos\frac{1}{x_1} + 10x_2^4\cos\frac{1}{x_2}  \right| \leqslant 14. \\
\left|\dfrac{\partial f}{\partial x_2}\right| &= \left| 40x_1x_2^3\cos\frac{1}{x_2} + 10x_1x_2^2\sin\frac{1}{x_2} \right| \leqslant 50.
\end{aligned}
$$

Далее выберем окрестность радиуса $\delta$: $B_\delta(x^*)=\{ x: ||x-x^*||\leqslant \delta \}$. \
Пусть $p$ - вероятность того, что в $B_\delta(x^*)$ находится хотя бы одна точка выборки.


\section{Задание 8}

\begin{enumerate}

\item Применить метод Монте-Карло к решению первой краевой задачи для двумерного уравнения Лапласа в единичном круге
$$
\begin{cases}
\Delta u = 0,(x,y)\in D \\
u\vert_{\delta D} = f(x,y) \\
u\in C^2(D), f\in C(\delta D) \\
D=\{(x,y)\in\mathbb{R}^2: x^2+y^2\leqslant 1 \}
\end{cases}
$$

\item Для функции $f(x,y) = x^2-y^2$ найти аналитическое решение и сравнить с полученным по методу Монте-Карло.

\end{enumerate}

\subsection{Метод Монте-Карло для уравнения Лапласа}

Будем решать задачу не в единичном круге, а в единичном квадрате. Решение при этом не изменится, ведь $x^2-y^2$ --- единственное решение поставленной задачи в единичном круге, которое является непрерывным. После работы алгоритма выберем только те точки, которые лежат в единичном круге.

Введём сетку на единичном квадрате $|x|\leqslant1, |y|\leqslant1$.

Конечно-разностная схема для уравнения Лапласа представима в виде
$$
u(P) = \dfrac{1}{4}\left( u(P_1) + u(P_2) + u(P_3) + u(P_4) \right).
$$

\subsection{Сравнение численного результата с аналитическим решением}

Функция $f(x,y)=x^2-y^2$ удовлетворяет $\Delta u=0$. Для внешней задачи Дирихле условием регулярности является ограниченность функции $u$. Функция $f=u$ является ограниченной, следовательно, данное уравнение имеет единственное решение, равное $x^2-y^2$.

\section{Задание 9}

\section{Задание 10}

\section{Задание 11}

\begin{thebibliography}{99}

\bibitem{buslenko}
\textit{Бусленко~Н.\,П., Шрейдер~Ю.\,А.}
Метод статистических испытаний, 1961, с.~117-122.

\bibitem{chist}
\textit{Чистяков~И.\,А.}
Лекции по оптимальному управлению.~2023-2024.

\bibitem{4authors}
\textit{Понтрягин~Л.\,С., Болтянский~В.\,Г., Гамерклидзе~Р.\,В., Мищенко~Е.\,Ф.}
Математическая теория оптимальных процессов.---4-е изд.---М.:~<<Наука>>,~1983.

\end{thebibliography}

\end{document}
